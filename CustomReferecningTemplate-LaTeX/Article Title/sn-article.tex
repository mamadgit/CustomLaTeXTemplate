%Version 3 October 2023
% See section 11 of the User Manual for version history
%
%%%%%%%%%%%%%%%%%%%%%%%%%%%%%%%%%%%%%%%%%%%%%%%%%%%%%%%%%%%%%%%%%%%%%%
%%                                                                 %%
%% Please do not use \input{...} to include other tex files.       %%
%% Submit your LaTeX manuscript as one .tex document.              %%
%%                                                                 %%
%% All additional figures and files should be attached             %%
%% separately and not embedded in the \TeX\ document itself.       %%
%%                                                                 %%
%%%%%%%%%%%%%%%%%%%%%%%%%%%%%%%%%%%%%%%%%%%%%%%%%%%%%%%%%%%%%%%%%%%%%

%%\documentclass[referee,sn-basic]{sn-jnl}% referee option is meant for double line spacing

%%=======================================================%%
%% to print line numbers in the margin use lineno option %%
%%=======================================================%%

%%\documentclass[lineno,sn-basic]{sn-jnl}% Basic Springer Nature Reference Style/Chemistry Reference Style

%%======================================================%%
%% to compile with pdflatex/xelatex use pdflatex option %%
%%======================================================%%

%%\documentclass[pdflatex,sn-basic]{sn-jnl}% Basic Springer Nature Reference Style/Chemistry Reference Style


%%Note: the following reference styles support Namedate and Numbered referencing. By default the style follows the most common style. To switch between the options you can add or remove “Numbered” in the optional parenthesis. 
%%The option is available for: sn-basic.bst, sn-vancouver.bst, sn-chicago.bst%  
 
%%\documentclass[sn-nature]{sn-jnl}% Style for submissions to Nature Portfolio journals
%\documentclass[sn-basic,Numbered]{sn-jnl}% Basic Springer Nature Reference Style/Chemistry Reference Style
%\documentclass[sn-mathphys-num]{sn-jnl}% Math and Physical Sciences Numbered Reference Style 
%%\documentclass[sn-mathphys-ay]{sn-jnl}% Math and Physical Sciences Author Year Reference Style
%%\documentclass[sn-aps]{sn-jnl}% American Physical Society (APS) Reference Style
%\documentclass[sn-vancouver,Numbered]{sn-jnl}% Vancouver Reference Style
%%\documentclass[sn-apa]{sn-jnl}% APA Reference Style 
%\documentclass[sn-chicago,Numbered, maxbibnames=20]{sn-jnl}% Chicago-based Humanities Reference Style
\documentclass{sn-jnl}
\usepackage[backend=biber, style=APA-custom, citestyle=numeric]{biblatex}  

%%%% Standard Packages
%%<additional latex packages if required can be included here>
\usepackage{graphicx}%
\usepackage{multirow}%
\usepackage{amsmath,amssymb,amsfonts}%
\usepackage{amsthm}%
\usepackage{mathrsfs}%
\usepackage[title]{appendix}%
\usepackage{xcolor}%
\usepackage{textcomp}%
\usepackage{manyfoot}%
\usepackage{booktabs}%
\usepackage{algorithm}%
\usepackage{algorithmicx}%
\usepackage{algpseudocode}%
\usepackage{listings}%
\usepackage{etoolbox}
\usepackage{calc}
\usepackage{xstring}
\usepackage{expl3}
%%%
%%%%%=============================================================================%%%%
%%%%  Remarks: This template is provided to aid authors with the preparation
%%%%  of original research articles intended for submission to journals published 
%%%%  by Springer Nature. The guidance has been prepared in partnership with 
%%%%  production teams to conform to Springer Nature technical requirements. 
%%%%  Editorial and presentation requirements differ among journal portfolios and 
%%%%  research disciplines. You may find sections in this template are irrelevant 
%%%%  to your work and are empowered to omit any such section if allowed by the 
%%%%  journal you intend to submit to. The submission guidelines and policies 
%%%%  of the journal take precedence. A detailed User Manual is available in the 
%%%%  template package for technical guidance.
%%%%%=============================================================================%%%%

%% as per the requirement new theorem styles can be included as shown below
\theoremstyle{thmstyleone}%
\newtheorem{theorem}{Theorem}%  meant for continuous numbers
%%\newtheorem{theorem}{Theorem}[section]% meant for sectionwise numbers
%% optional argument [theorem] produces theorem numbering sequence instead of independent numbers for Proposition
\newtheorem{proposition}[theorem]{Proposition}% 
%%\newtheorem{proposition}{Proposition}% to get separate numbers for theorem and proposition etc.

\theoremstyle{thmstyletwo}%
\newtheorem{example}{Example}%
\newtheorem{remark}{Remark}%

\theoremstyle{thmstylethree}%
\newtheorem{definition}{Definition}%

\raggedbottom
%%\unnumbered% uncomment this for unnumbered level heads
\addbibresource{bibliography.bib}
\begin{document}
\title[Article Title]{Article Title}

%%=============================================================%%
%% GivenName	-> \fnm{Joergen W.}
%% Particle	-> \spfx{van der} -> surname prefix
%% FamilyName	-> \sur{Ploeg}
%% Suffix	-> \sfx{IV}
%% \author*[1,2]{\fnm{Joergen W.} \spfx{van der} \sur{Ploeg} 
%%  \sfx{IV}}\email{iauthor@gmail.com}
%%=============================================================%%

\author*[1]{\fnm{Amir Mohammad} \sur{Tahsiri}}\email{mohammad\_tahsiri@hotmail.com}

% \author[2,3]{\fnm{Second} \sur{Author}}\email{iiauthor@gmail.com}
% \equalcont{These authors contributed equally to this work.}

% \author[1,2]{\fnm{Third} \sur{Author}}\email{iiiauthor@gmail.com}
% \equalcont{These authors contributed equally to this work.}

\affil*[1]{\orgaddress{\street{14th}, \city{Tehran}, \postcode{100190}, \state{Tehran}, \country{Iran}}}

% \affil[2]{\orgdiv{Department}, \orgname{Organization}, \orgaddress{\street{Street}, \city{City}, \postcode{10587}, \state{State}, \country{Country}}}

% \affil[3]{\orgdiv{Department}, \orgname{Organization}, \orgaddress{\street{Street}, \city{City}, \postcode{610101}, \state{State}, \country{Country}}}

%%==================================%%
%% Sample for unstructured abstract %%
%%==================================%%

\abstract{In the context of multi-agent systems, formal modelling is considered a powerful method to capture an intrinsic representation of ``behaviour'' and ``belief'' in autonomous and intelligent agents. Various semantical models can be used to reason about agents' (state of) knowledge/beliefs to capture this representation. Many of these models often seek to enrich traditional (epistemic) models by computationally defining agents' reasoning processes, thereby incorporating computationally grounded criteria. However, by reviewing past research and analysing established semantics in theory and application, this paper will demonstrate that the concept of computationally grounded semantics is not fixed (i.e., criteria to be met), but rather exists on a continuum or a gradient. This is a significant notion to consider, as it suggests that the level of computational detail necessary to ground the design of a semantical model, so it may intrinsically represent agents' behaviour and beliefs, should be proportional to its environment. This means in certain environments, its even possible a more abstract and flexible design may be required. The relationship between the computational details of a semantical model and its environment is a topic that previous research has not thoroughly explored.\par
The current research landscape in formally modelling agent behaviour to create a computationally grounded semantical model has primarily involved either enhancing traditional epistemic models by including more computational details or developing entirely new types of formal models with different reasoning processes (such as temporal, stochastic, concurrent, etc., structures). According to those research, once specific criteria are satisfied, the semantics are deemed to be computationally grounded.\par
To elaborate on the importance of this perspective, this paper will present a semantical model known as concurrent game structures where the agents are resource-bounded as an example. While this model is considered an excellent example with solid computational details, this paper will argue that in the pursuit of computationally grounded semantics, one potential problem is that the model in a sense sacrifices the agents' autonomy. Such semantics makes it difficult to trace the reasoning process of any singular agent meaning an individual agent cannot explain its reasoning process. Hence, there is a need to consider proportionality when designing semantical structures and their environment. To that point, this paper will suggest a new route in which semantical models are computationally grounded (with relatively high level of computational details), but where autonomous agents can individually explain their reasoning process.}

%%================================%%
%% Sample for structured abstract %%
%%================================%%

% \abstract{\textbf{Purpose:} The abstract serves both as a general introduction to the topic and as a brief, non-technical summary of the main results and their implications. The abstract must not include subheadings (unless expressly permitted in the journal's Instructions to Authors), equations or citations. As a guide the abstract should not exceed 200 words. Most journals do not set a hard limit however authors are advised to check the author instructions for the journal they are submitting to.
% 
% \textbf{Methods:} The abstract serves both as a general introduction to the topic and as a brief, non-technical summary of the main results and their implications. The abstract must not include subheadings (unless expressly permitted in the journal's Instructions to Authors), equations or citations. As a guide the abstract should not exceed 200 words. Most journals do not set a hard limit however authors are advised to check the author instructions for the journal they are submitting to.
% 
% \textbf{Results:} The abstract serves both as a general introduction to the topic and as a brief, non-technical summary of the main results and their implications. The abstract must not include subheadings (unless expressly permitted in the journal's Instructions to Authors), equations or citations. As a guide the abstract should not exceed 200 words. Most journals do not set a hard limit however authors are advised to check the author instructions for the journal they are submitting to.
% 
% \textbf{Conclusion:} The abstract serves both as a general introduction to the topic and as a brief, non-technical summary of the main results and their implications. The abstract must not include subheadings (unless expressly permitted in the journal's Instructions to Authors), equations or citations. As a guide the abstract should not exceed 200 words. Most journals do not set a hard limit however authors are advised to check the author instructions for the journal they are submitting to.}

\keywords{keyword1, Keyword2, Keyword3, Keyword4}

%%\pacs[JEL Classification]{D8, H51}

%%\pacs[MSC Classification]{35A01, 65L10, 65L12, 65L20, 65L70}

\maketitle

\section{Introduction}\label{sec1}

The \emph{Belief-Desire-Intention} (BDI) framework with semantics based on possible worlds is prevalent and effective for reasoning about agent behaviour regarding the epistemic knowledge of autonomous and intelligent agents in multi-agent systems. Since it is somewhat difficult to determine how intelligent agents reason internally, the possible worlds semantics offers a solution by focusing on the agents' actions and its effects on the environment to explain the agents' reasoning process. The possible worlds semantics claims that an agent's knowledge state is determined by the states of the overall system and its environment, collectively referred to as global states (the technical definition of possible worlds semantics will be elaborated in section two). Much of the research regarding reasoning about agent behaviour has been centred around this framework and semantics \cite{10.1007/3-540-49057-4_1}. However, a significant problem that still lies around the BDI framework is the issue of agents being modelled as logically omniscient, which is not a realistic property as it suggests that an agent knows all the consequences of its knowledge \cite{DBLP:books/mit/FHMV1995, alma990002228290206881}. This issue arises in standard epistemic logics and frameworks like BDI because it is unclear how to align the computational implementations of an agent with its knowledge. \cite{DBLP:conf/mallow/AlechinaL10} This misalignment results in the semantics of epistemic logics, lacking the property of being \emph{computationally grounded}. The concept of computationally grounded semantics will be discussed in detail further. \par 

\printbibliography% common bib file
%% if required, the content of .bbl file can be included here once bbl is generated
%%\input sn-article.bbl

\end{document}
